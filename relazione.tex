% Marco l'Eccellente Dio della Modestia
% !TeX encoding = utf8
% !TeX program = pdflatex
% !TeXpellcheck = it_IT

\documentclass[a4paper,11pt,oneside]{article} 
\usepackage{relazioni}
\usepackage{imakeidx}
\usepackage{colortbl}
\usepackage{booktabs}
\usepackage{blindtext}
\usepackage{titletoc}
\usepackage{hyperref}
\usepackage{graphicx}
\usepackage{subcaption}
\usepackage{wrapfig}
\usepackage{subfig}
\usepackage{geometry}
\usepackage{array}
\usepackage{multirow}
\usepackage{multicol}
\usepackage{rccol}
\usepackage{amssymb}



\usepackage{hyperref}

\usepackage[export]{adjustbox}
\usepackage[export]{adjustbox}
\hypersetup{
%    colorlinks=false,
} 

\graphicspath{{Figure/}} 
%https://www.overleaf.com/learn/latex/Indices
%\makeindex[columns=3, title=Alphabetical Index, intoc]

\setlength{\parindent}{0em}

\begin{document}
\input{Front-matter/Frontespizio}
\clearpage
\tableofcontents
\addtocontents{toc}{~\hfill{Pagina}\par}
\contentsmargin{6em}
\dottedcontents{section}[1em]{\bigskip}{2em}{1pc}
\dottedcontents{subsection}[3em]{\smallskip}{3em}{1pc}
\dottedcontents{subsubsection}[5em]{\smallskip}{4em}{1pc}


\newpage


\section{Obiettivo esperienza}
L'obiettivo dell'esperienza è la caratterizzazione

\section{Apparato Sperimentale}
\begin{figure}[h!]
    \centering
    \includegraphics[width=10cm]{ApparatoSperimentale.pdf}
    \caption{Apparato Sperimentale}
    \label{fig:apparato_sperimentale}
\end{figure}

L'apparato sperimentale è composto dai seguenti elementi: 
\begin{enumerate}[label=\textbf{\alph*.}]
    \item Sensori/ strumentazione di acquisizione dati
    \item Valvola a \textbf{T}
    \item Sensore di temperatura
    \item Siringa monouso da $\SI{20}{\centi\meter\cubed}$ per compiere le compressioni ed espansioni
    \item Resistenza elettrica
    \item Agitatore meccanico
    \item Tubicino di plastica immerso nel bagno termico e collegato ai sensori
    \item Bagno termico con acqua e inizialmente ghiaccio
    \item Contenitore Dewar isolante
    \item Pareti adiabatiche isolanti
\end{enumerate}

\section{Presa dati}
Tutti i dati acquisiti ad intervalli di tempo regolari sono stati forniti dalla strumentazione elettronica dell'apparato.Per ogni misurazione si sono registrati i valori di pressione, temperatura e volume. Quest'ultimo è stato calcolato dal software a partire dalla posizione dello stantuffo della siringa, poi moltiplicata per la sezione della stessa ma non considerando nel computo il volume di gas contenuto nei tubicini, in quanto superfluo ai fini dell'esperienza. Nonostante ciascun sensore, rispetto agli altri, sia indipendente nell'acquisizione, è comunque possibile che si crei una correlazione nel momento dell'acquisizione del segnale elettrico a seconda di come i circuiti sono tra loro collegati.\newline

Si sono realizzate alcune prese dati variando di volta in volta la temperatura dell'acqua nella quale il sistema era immerso, cercando di mantenere il più possibile ogni dispersione termica, tramite l'impiego del vaso Dewar. Con il contenitore inizialmente a temperatura prossima agli $\SI{0}{\degree}$ centigradi, per via del ghiaccio presente al suo interno, si è effettuata una prima presa dati. Successivamente si è aumentata la temperatura del bagno termico, misurata dal sensore, tramite la resistenza immersa di $\approx\SI{10}{\degree}$. Assestatasi la temperatura al valore prefissato si è rieseguita la presa dati. Tale processo si è ripetuto per tutte le temperature riportate in Tabella \ref{tab:camp_temp}. Si specifica che la presa dati si è svolta muovendo opportunamente una manovella in modo da controllare la compressione e la decompressione della siringa, mescolando occasionalmente il liquido all'interno del Dewer tramite l'agitatore. Si noti che i due campioni finali realizzati nel momento in cui la temperatura era ancora a $\SI{25}{\celsius}$ gradi si differenziano per modalità di presa dati, rispettivamente con una variazione più lenta e con una variazione più veloce rispetto a quelle mediamente usate per tutti gli altri campioni.

\begin{table}[h!]
    \centering
    \begin{tabular}{|c|c|c|c|c|c|c|c|c|}
        \hline
        Campione & $1$ & $2$ & $3$ & $4$ & $5$ & $6$ & $3_{lenta}$ & $3_{veloce}$\\\hline
        \rowcolor[rgb]{0.85,0.85,0.85} T [$\si{\celsius}$] & $\approx5$ & $\approx15$ & $\approx25$ & $\approx35$ & $\approx45$ & $\approx55$ & $\approx25$ & $\approx25$ \\\hline
    \end{tabular}
    \caption{Temperature dei campioni}
    \label{tab:camp_temp}
\end{table}

%Tabella campioni e temperature approssimate, solo per dare un idea tra numero campione e temperatura


\section{Analisi e Discussione}
\subsection{Fasi compressione e decompressione}\label{par:compress}
%separazione fasi compressione e decompressione
%   - perchè si separano? che motivi ci sono
Come operazione preliminare è stato necessario suddividere i dati a seconda della fase di compressione o decompressione a cui essi appartengono. Questa scelta è stata necessaria per verificare successivamente se le due fasi fossero tra loro compatibili, ovvero che queste trasformazioni fossero approssimativamente reversibili e quasi statiche.

Per dividere opportunamente i dati in sottocampioni, da ora in poi chiamati campioni in compressione e decompressione, si sono osservati i grafici dei dati grezzi. Si è osservato che i valori della pressione nella prima fase del ciclo aumentano in funzione del numero della misurazione, per poi assestarsi e rimanere costanti prima dell'inizio della fase di decompressione. Sebbene ci siano delle piccole variazioni di circa $\SI{0.01}{\centi\meter\cubed}$ nel range di misurazioni in cui si osserva un volume costante, si può affermare con certezza che proprio in questa fase cambia il verso di trasformazione, da compressione si inizia a decomprimere il gas. Si è pertanto scelto di considerare il numero della misurazione intermedio del range nel quale la pressione risultava pressoché costante per suddividere i dati grezzi. Tale scelta ha permesso di separare le due fasi anche per le altre grandezze misurate. Per tale considerazione non si sono utilizzate le informazioni date dall'andamento del volume in quanto soggetto a maggiori variazioni nel range in cui invece la pressione rimaneva costante. Si ipotizza che questa variabilità del volume sia dovuta alla presenza di uno stantuffo di gomma nella siringa, utilizzato come guaina a tenuta stagna per prevenire fuoriuscite di gas, che varia il suo volume in quanto costituito da materiale comprimibile, ma non la pressione in quanto non altera significativamente il volume del gas.

%come capire per tagliare decompress e compress
\begin{figure}[h!]
    \centering
    \includegraphics[width=10cm]{Pressione_Volume.pdf}
    \caption{Pressioni e Volumi Primo Campione}
    \label{fig:campione2}
\end{figure}



\subsection{Stime della temperatura}
%stima temperatura
%   - commento perchè varia nonostante debba essere costante perchè isolata
%   - non fisictà di picco
%   - reiezione ad occhio dal picco
%   - spiegazione perchè scende (non completamente adiabatico il contenitore, cerca equilibrio con ambiente)
%   - media è quella più appropiata
Una prima analisi dei dati relativi alle temperature mostrano talvolta una leggera variazione di questi ultimi durante tutte e due le fasi della trasformazione. Si osserva infatti che dispersione massima di $\approx \SI{1.4}{\kelvin}$, soprattutto nei campioni 2, 3, 5 e 6. 

Si riporta, a titolo di esempio per tutti i campioni in cui si verifica questo fenomeno di variazione iniziale della temperatura, il Grafico \ref{fig:campione2}. Si può osservare che nelle prime misurazioni la temperatura non rimane costante, bensì dapprima diminuisce, poi aumentare per assestarsi infine al valore asintotico. La variazione è di poco meno di $\SI{2}{\kelvin}$, poco significativa se messa in confronto con il valore a cui tende. %intendo err percentuale piccolo
Si ipotizza che queste variazioni, che si verificano anche per altri campioni e soltanto nelle prime misurazioni siano attribuibili ad un rimescolamento del liquido da parte dell'operatore. È ragionevole che il liquido presenti differenze di temperatura dell'ordine $1-2$ gradi Kelvin proprio per come è costruito il sistema e che il rimescolamento del liquido porti ad una variazione della temperatura misurata dalla termocoppia. 
Si è deciso comunque di stimare la temperatura dell'isoterma come media aritmetica delle temperature misurate per tutti i campioni, avendo cura però di non considerare i valori precedenti ad una certa soglia arbitrariamente fissata per i campioni che presentano un picco, ovvero un cambiamento anomalo. Ad esempio, nel secondo campione la media è stata calcolata a partire dalla 300esima misura. Gli errori assegnati alle stime delle temperature sono derivati dalla deviazione standard sulla media, motivo per cui gli errori percentuali sono di molto inferiori a 1\%.

Specificatamente per il secondo campione, si nota la presenza di alcuni punti attorno alla 800esima misurazione che si discostano fortemente dalle temperature circostanti. Ciò è probabilmente imputabile ad un errore di lettura del sensore. Nonostante ciò si è deciso di non scartare tali valori.

Altra osservazione derivata da un'attenta osservazione dei grafici, è che se la temperatura impostata del gas risulta maggiore della temperatura dell'ambiente circostante, stimata a circa $\SI{20}{\celsius}$, allora la prima tende a diminuire per raggiungere l'equilibrio termico con l'ambiente esterno. Seppur contenitore isolante, vi sono comunque dispersioni di calore dovute alle pareti non perfettamente isolanti, alla parte di tubicino di plastica che fuoriesce dal tappo per connettersi ai sensori e la parte di siringa non immersa nel fluido. Per questa ragione si è stimata un'unica temperatura sia per la fase in compressione che per quella in decompressione che vengono riportate nella Tabella \ref{tab:temp_unica}. 

\begin{table}[h!]
    \centering
    \begin{tabular}{|c|c|}
        \hline
        \textbf{Campione} & \textbf{T} [$\si{\kelvin}$] \\ \hline
        \rowcolor[rgb]{0.85,0.85,0.85}$1$ & $279.0630\pm0.0009$ \\ \hline
        $2$ & $288.7700\pm0.0005$ \\ \hline
        \rowcolor[rgb]{0.85,0.85,0.85}$3$ & $298.1470\pm0.0005$ \\ \hline
        $4$ & $308.251\pm0.001$ \\ \hline
        \rowcolor[rgb]{0.85,0.85,0.85}$5$ & $317.593\pm0.003$ \\ \hline
        $6$ & $327.084\pm0.008$ \\ \hline \hline
        \rowcolor[rgb]{0.85,0.85,0.85}$3$ & $298.1470\pm0.0005$ \\ \hline
        $3_{lenta}$ & $298.3070\pm0.0006$ \\ \hline
        \rowcolor[rgb]{0.85,0.85,0.85}$3_{veloce}$ & $298.359\pm0.001$ \\ \hline
    \end{tabular}
    \caption{Temperature uniche per ciascun campione}
    \label{tab:temp_unica}
\end{table}


%grafico dettagli per picco temperature
\begin{figure}[h!]
    \centering
    \includegraphics[width=10cm]{temperatura_secondo.pdf}
    \caption{Temperature secondo campione}
    \label{fig:campione2}
\end{figure}

%precednetenetemenete
\subsection{Analisi delle fasi di compressione e decompressione}
Un'osservazione generale dei dati ha mostrato che nella parte iniziale del ciclo, corrispondente alla prima parte della fase di compressione, il volume rimane costante per circa 30 misurazioni per poi diminuire nel modo atteso. Poiché durante la presa dati la manovella è stata girata in maniera costante, ci si aspetterebbe una variazione di volume anche in questa prima fase. Per interpretare questo fenomeno, si è ipotizzato che nella fase iniziale ci siano stati dei giri a vuoto della manovella o del meccanismo che la muove.
Si è proceduto pertanto alla reiezione di tali dati eseguita secondo il procedimento qui descritto.
Analogamente, un fenomeno simile accadeva per la pressione. Al massimo della compressione infatti la pressione rimaneva costante, come già descritto nel paragrafo \ref{par:compress}. Si è pertanto deciso di scartare le terne appartenenti all'intervallo in cui la pressione rimaneva costante in quanto non facenti parte della trasformazione isoterma.
Per ciascun campione, ciascuna terna i-esima di pressione, volume e temperatura è stata dunque sottoposta ad un duplice controllo su pressione e volume. Talora i valori indicassero una appartenenza della terna al range della trasformazione non definito isotermo, essi venivano scartati. Questa procedura è stata eseguita sia sulle misure di compressione che decompressione.

TABELLA NUMERO MISURE SCARTATE
%inserire il numero degli indici scartati




%reiezione pressione
% analisi su scarti cosa rivela? viene più preciso? DA FARE

Si è proceduto alla rappresentazione grafica delle grandezze pressione e volume adeguatamente separate per la fase di compressione e decompressione. Nello specifico sono stati realizzati i grafici utilizzando sull'ascissa il valore di 1/P e sull'asse delle ordinate il valore del rispettivo volume misurato. Tale scelta è stata determinata dalla considerazione dell'equazione di stato dei gas perfetti:
\begin{gather*}
    PV_{tot} = n R T \\
    V_{tot}= n R T \cdot \frac{1}{P}\\
    V_{tot}=V_{siringa}+V_{tubicini}=n R T \frac{1}{P}\\
    V_{siringa}=n R T \frac{1}{P} - V_{tubicini}
\end{gather*}
assumendo che la trasformazione compiuta durante la presa dati sia isoterma. La rappresentazione dei dati come prima specificato, risulta dunque giustificata al fine di testare l'ipotesi di una dipendenza lineare della volume V in funzione di 1/P e successivamente che la trasformazione sia reversibile.
Vengono di seguito riportati i 6 grafici 1/P vs. V relativi a ciascun campione con le interpolazioni distinte per fase della trasformazione e le rette interpolante.

\begin{figure}
    \centering
    \makebox[\textwidth]{
    \subfloat[Temperature primo campione]{
        \includegraphics[width=7cm]{Zero_Clapeyron.pdf}
        \label{fig:campionezero_clapeyron}
    }
    \subfloat[Temperature secondo campione]{
        \includegraphics[width=7cm]{Uno_Clapeyron.pdf}
        \label{fig:campioneuno_clapeyron}
    }}
    \newline
    \makebox[\textwidth]{
    \subfloat[Temperature terzo campione]{
        \includegraphics[width=7cm]{Due_Clapeyron.pdf}
        \label{fig:campionedue_clapeyron}
    }
    \subfloat[Temperature quarto campione]{
        \includegraphics[width=7cm]{Tre_Clapeyron.pdf}
        \label{fig:campionetre_clapeyron}
    }}
    \newline
    \makebox[\textwidth]{
    \subfloat[Temperature quinto campione]{
        \includegraphics[width=7cm]{Quattro_Clapeyron.pdf}
        \label{fig:campionequattro_clapeyron}
    }
    \subfloat[Temperature sesto campione]{
        \includegraphics[width=7cm]{Cinque_Clapeyron.pdf}
        \label{fig:campionecinque_clapeyron}
    }}
    \label{fig:my_label}
\end{figure}

I grafici generalmente rispettano un andamento concorde con le aspettative teoriche. Il fit lineare effettuato per le due fasi, i cui parametri vengono riportati nei grafici, ha permesso di confrontare le fasi di compressione e decompressione tramite la comparazione dei parametri ottenuti.

Per verificare la reversibilità delle due trasformazioni si è valutata la compatibilità fra i due coefficienti angolari delle rette interpolanti. Le attese teoriche prevedono infatti che il valore del coefficiente angolare ottenuto, corrispondente dunque a $n \cdot R \cdot T$ nelle due fasi sia compatibile. Per valutare la compatibilità si è fatto affidamento sulla seguente 

e calcolando $\lambda$ tra i coefficienti angolari di ogni campione si è riscontrata un valore almeno discreto, ad eccezione del primo ed il quarto campione. Per questi ultimi infatti si è ricavato un valore di $\lambda$ che mostra una non compatibilità delle due differenti fasi. La spiegazione di tale incompatibilità dei coefficienti angolari tra le due differenti fasi è imputabile alle modalità di presa dati. Qualsiasi la causa, l'incompatibilità suggerisce perlomeno che la trasformazione non sia stata del tutto reversibile o quasi statica. Come atteso il coefficiente angolare aumenta in funzione della temperatura, come si può dedurre dall'equazione sopra riportata.

\begin{table}[h!]
\centering
\begin{tabular}{|c|c|c|c|c|c|c|}
\hline
\textbf{Campione} & $0$ & $1$ & $2$ & $3$ & $4$ & $5$ \\ \hline
$\lambda_{b_{comp},b_{decomp}}$ & $7.1$ & $0.9$ & $0.2$ & $8.6$ & $1.0$ & $0.1$ \\ \hline
\end{tabular}
\end{table}

Focalizzando poi l'attenzione sulle intercette, si osserva che nei vari grafici tali valori, che dovrebbe rimanere costante in tutti i campioni considerati, tende ad aumentare con l'aumentare della temperatura e si osserva inoltre che nelle due differenti fasi di compressione e di decompressione il valore dell'intercetta sia  differente. La spiegazione di questo secondo fenomeno è da ricercarsi nel non perfetto isolamento termico del Dewer. Infatti, a causa della perdita di calore del serbatoio non perfettamente adiabatico, si ha una diminuzione della temperatura che, conseguentemente, determina una piccola riduzione del volume, manifestata nella differenza del valore dell'intercetta calcolato. Sapendo che la fase di decompressione è stata eseguita successivamente alla fase di compressione, si ha che il valore dell'intercetta sia più grande nella fase di decompressione piuttosto che nella fase di compressione. La spiegazione del primo fenomeno descritto invece potrebbe essere legata al materiale plastico con il quale sono stati costruiti i tubicini e alla dilatazione termina di tale materiale. Infatti all'aumentare della temperatura aumenta conseguentemente il valore dell'intercetta calcolato e dunque il volume del gas presente all'interno dei tubicini.


%chi quadro
Per quanto riguarda il valore ottenuto effettuando il test del $\chi^2$, si ha che il rapporto $\chi^2/GDL$ è sempre inferiore ad 1, dunque si può affermare che l'ipotesi di linearità viene accettata con un livello di confidenza del 99,5\%.
In alcuni campioni però 


%problema del piccolo saltino del volume a 8.85
\begin{figure}[h!]
    \centering
    \includegraphics[width=8cm]{problma_volume.pdf}
    \caption{Zoom volume}
    \label{fig:problema_volume}
\end{figure}



%analisi compressionene e decompressione, quale funzione dovrebbero seguire
% ipotesi che è gas ideale?
%   - come scartare i dati, solo quando sono costanti i volumi e le pressioni
%   - spiegazione perchè li elimini, non è fisico, errore della siringa  e acquisizione dati, quali dati e quanti dati sono stati rimossi
%   - analisi su scarti cosa rivela? viene più preciso? DA FARE
%   - equazione rivela volume tubicini, non sono immersi, disperdono calore, anche la siringa non è in bagno termico
%   - chi quaro gigantesco perchè? in confronot a gdl okay. anche se metti tutti i dati è okay-> non ci si può basare solo su chi quadro per verificare se il fit è okay
%   - interpolazione lineare cosa indicano i vari coefficienti, stima err posteriori e confronto con quello a priori,
%   - rho e tstudent su rho commenti
%   - compressione è quasi statica? compatibilità fra coeff ang compress e decompress
%   - si osserva che aumentando il campione aumenta il coeff angolare, banana deve essere così perchè dipende direttamente dalla temperatura
%    - stima di numero di moli per ciascun fit in compressione e decompressione


%analisi di tutti i coeff angolari in compressione e decompressione su stesso grafico in funzione di temperatura stimata
%   - tutta la algebretta del cazzo che mostra come trovare lo zero assoluto
%   - stima del valore di zero assoluto e commento a suo errore
%   - perchè alcuni dati sono fuori dal fit e non vicini a quello che ci si aspetta-> errore in coeff angolare o temperatura?
%    - stima moli da fit generale per vedere se sensato e compatibili con stime precedenti

\subsection{Stima delle moli campionarie}
%stima di numero moli da tutte le triple p v e t e si vede che in questo caso sono minori perché è solo quello dentro la siringa
Si sono inoltre calcolate le moli considerando al media fra tutti i singoli valori di pressione, volume e temperatura per ciascuna isoterma, ottenendo un campione di valori da cui poi si è stimata la media. Infine da tutte le stime derivate dalle single isoterme si è ricavato il numero di moli riassuntivo con il relativo errore sulla media.
\begin{gather*}
    \left \{n_{i, j}=\frac{P_{i, j}V_{i, j}}{T_{i, j}R}\right \}_{i, j} \Rightarrow \overline{n_{i, j}} \hspace{1cm}\forall \text{campione j-esimo}\\
    \{\overline{n_{i}}\}_{j} \Rightarrow \overline{n}\pm \sigma_{\overline{n}}
\end{gather*}


\begin{wraptable}{r}{5cm}
    \begin{tabular}{|c|c|}
        \hline
        \multirow{2}{*}{\textbf{Campione}} & \textbf{Moli campionarie}\\
        & $10^{-6}$ [mol] \\ \hline
        $1$ & $921$ \\ \hline
        $2$ & $918$ \\ \hline
        $3$ & $924$ \\ \hline
        $4$ & $907$ \\ \hline
        $5$ & $897$ \\ \hline
        $6$ & $912$ \\ \hline
        $\overline{n}\pm\sigma_{\overline{n}}$ & $913 \pm 4$ \\ \hline
    \end{tabular}
    \caption{Moli Campionarie}
    \label{tab:moli_campionarie}
\end{wraptable}

I valori ottenuti sono riportati in Tabella \ref{tab:moli_campionarie} da cui è immediato osservare che essi risultano sempre sottostimati rispetto a quelli calcolati tramite l'interpolazione lineare in quanto la stima campionaria si riferisce solo alle moli contenute nella siringa, quella tramite fit alle moli totali.

Come accennato precedentemente non è possibile escludere una correlazione fra le tre grandezze riferite ad una stessa misurazione, né a misure consecutive nello stesso campione di una singola isoterma. La possibile correlazione può essere ricondotta al metodo di acquisizione dati dei circuiti impiegati. In questa analisi tuttavia, essendo dubbia l'entità e l'importanza di questa correlazione, tutti i dati sono stati considerati scorrelati. Tuttavia, proprio per scongiurare questo rischio, si è optato per stimare $\overline{n}$ tramite una media aritmetica non una media ponderata, associandovi in ogni caso la $\sigma_{\overline{n}}$. L'errore percentuale di $\overline{n}$ risulta essere di $\approx 0.4 \%$, valore piuttosto piccolo, probabilmente sottostimato. 




\subsection{Simulazione di isocora}
Partendo dai dati forniti è stata simulata un'isocora: si sono cercati inizialmente un valore di volume in maniera che tale valore fosse presente in tutti i 6 campioni. Si è deciso di trovare 2 volumi al fine di poter confrontare i risultati ottenuti. Questi volumi sono $8.63$ $cm^3$ e $21.01$ $cm^3$.
A partire dalle terne di dati relative ai volumi analizzati sono stati realizzati i seguenti grafici.

\begin{figure}[h!]
    \centering
    \makebox[\textwidth]{
    \subfloat[Prima isocora]{
        \includegraphics[width=8.75cm]{Isocora_8-63.pdf}
        \label{fig:prima_isocora}
    }
    \subfloat[Seconda isocora]{
        \includegraphics[width=8.75cm]{Isocora_21-01.pdf}
        \label{fig:seconda_isocora}
    }}
    \label{fig:isocore}
\end{figure}
I parametri del fit sono stati calcolati tramite il metodo del minimo $\chi^2$. Con 4 gradi di libertà, l'ipotesi di un andamento lineare è accettata con un CL del $99.5\%$

\subsection{Valutazione di quasi-staticità e reversibilità}
%valutazione quasi-staticità di compressioni e decompressioni
%   - compatibilità coeff angolari per comp decomp per tutte e 3
%   - verifica quantitativa di differenza fra i 3 campioni-> lento compatte intorno a retta, normale è un pò separato, veloce sono separate

\begin{figure}[h!]
    \centering
    \makebox[\textwidth]{
    \subfloat[Compressione]{
        \includegraphics[width=7cm]{entrambi_compressione.pdf}
        \label{fig:compressione}
    }
    \subfloat[Decompressione]{
        \includegraphics[width=7cm]{entrambi_decompressione.pdf}
        \label{fig:decompressione}
    }}
    \caption{Caption}
    \label{fig:my_label}
\end{figure}


\section{Conclusioni}

\newpage
\section{Appendice}
\subsection{Ulteriori grafici a completamento di quelli riportati in Analisi e Discussione}
\subsection{Formulario}
\textbf{Media, deviazione standard, deviazione standard della media}
\begin{align*}
   % \begin{aligned}
        \overline{x}&=\sum\limits_{i=1}^{N} \frac{x_{i}}{N}&
        \sigma&=\sqrt{\frac{\sum\limits_{i=1}^{N} (x_{i}-\overline{x})^2}{N-1}}&
        \sigma_{\overline{x}}&=\frac{\sigma}{\sqrt{N}}
   % \end{aligned}
\end{align*}\\

\textbf{Media Ponderata}
\begin{equation*}
\label{eq:media_pond}
    x_i=\frac{\sum_{i=1}^{N}\frac{x_i}{\sigma_{x_i}}}{\sum_{i=1}^{N}\frac{1}{\sigma_{x_i}}}
\end{equation*}

\textbf{Errore Media Ponderata}
\begin{equation*}
\label{eq:errore_media_pond}
     \sigma_{x_i}=\sqrt{\frac{1}{\sum_{i=1}^{N}\frac{1}{\sigma_{i}^{2}}}}
\end{equation*}

\textbf{Formule per il metodo del minimo ${\chi}^{(2)}$}
\begin{equation*}
        \begin{cases}
    a=&\frac{1}{\Delta}[(\sum\limits_{i=1}^{N}{x_{i}^{2}})\cdot(\sum\limits_{i=1}^{N}{y_{i}})-(\sum\limits_{i=1}^{N}{x_{i}})\cdot(\sum\limits_{i=1}^{N}{x_{i}y_{i}})] \\ 
    b=&\frac{1}{\Delta }\cdot \left [N\cdot \left ( \sum\limits_{i=1}^{N}x_i y_i \right )-\left ( \sum\limits_{i=1}^{N}x_i \right )\cdot \left ( \sum\limits_{i=1}^{N}y_i \right )  \right ]\\
    \Delta=& N\cdot \sum\limits_{i=1}^{N} x_i^{2} - \left ( \sum\limits_{i=1}^{N}x_i \right )^{2}\\
    \end{cases}
\end{equation*}
\begin{equation*}
    \begin{cases}
    \sigma_{a}=&\sigma_{y}\cdot\sqrt{\frac{\sum_{i=1}^{N}{x_{i}^{2}}}{\Delta}} \\
    \sigma_{b}=&\sigma_y\cdot \sqrt{\frac{N}{\Delta }}\\
    \end{cases}
    \label{equation:err_chi_quadro}
\end{equation*}
\\
\newline
\textbf{Formula di propagazione degli errori casuali}\\

Sia z=($x_1$,...;$x_N$) funzione di N variabili casuali $x_1$,...,$x_N$ e sia ${x_i^\ast}$=($x_1^\ast$,...,$x_N^{\ast}$) l'insieme di tutti i valori veri associati a tali variabili, si ha 

\begin{equation*}
    \sigma_z^{2}\approx  \sum_{i=j=1}^{N}\left ( \frac{\partial z}{\partial x_i}\Big|_{x_i^{\ast}} \right )^{2}\cdot\sigma_{x_i}^{2} +\sum_{i=1,j=1,i\neq j}^{N}\left (\frac{\partial z }{\partial x_i}\Big|_{x_i^{\ast}} \right ) \cdot \left ( \frac{\partial z}{\partial x_j} \Big|_{x_j^{\ast}} \right )\cdot cov(x_i,x_j)\label{eq:prop_errori}
\end{equation*}
E' stato utilizzato il simbolo $\approx$ in quanto si è scelto di troncare al primo termine lo sviluppo in serie di Taylor.\\


\textbf{Formula calcolo compatibilità}\\
\begin{equation*}
    \lambda=\frac{\left|a-b\right|}{\sqrt{\sigma^{2}_{a}+\sigma^{2}_{b}}}
\end{equation*}\\
\textbf{Coefficiente di correlazione di Pearson}\\
\begin{equation*}
    \rho=  \frac{\sum_{i=1}^{N}(x_i - \overline{x}
    )(y_i - \overline{y})}{\sqrt{\sum_{i=1}^{N}(x_i -\overline{x})^2}\sqrt{\sum_{i=1}^{N}(y_i - \overline{y})^2}}
\end{equation*}\textbf{
}

\section{Codici sorgente}
%\subsection{Programmi}
Si riportano i link ai codici sorgente impiegati per l'analisi.
\begin{itemize}
    \item \href{https://github.com/badbigota/6_relazione/blob/master/Programmi/analisi.cxx}{analisi.cxx} per la vera e propria analisi dati 
    \item \href{https://github.com/badbigota/6_relazione/blob/master/Programmi/functions.h}{functions.h} per gli algoritmi usati
    \item \href{https://github.com/badbigota/6_relazione/blob/master/Programmi/statistica.h}{statistica.h} per le funzioni statistiche impiegate
    \item\href{https://github.com/badbigota/6_relazione/blob/master/Programmi/struct.h}{struct.h} per la definizione delle strutture di dati impiegate
\end{itemize}

\end{document}
